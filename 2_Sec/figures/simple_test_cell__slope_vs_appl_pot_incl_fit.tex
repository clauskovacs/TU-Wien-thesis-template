% FIG:	COMSOL simulations: time-dependent study with varying applied potential at WE:
%
% extracted values: applied potential vs. slope
%
\begin{tikzpicture}[scale = \pScaleImgFactorDblImage*0.95, trim axis left, trim axis right]
	\begin{axis}[
 					xmin  = -0.8,
 					xmax  = -0.6,
   					xtick = {-0.80, -0.76,...,-0.6},
					xmajorgrids,
					xlabel = {Applied potential at WE [\SI{}{\volt}]},
 					ymin  = -50.00,
 					ymax  = 200.00,
%  					ytick = {-40, 0, 40, 80},
					ylabel = {Slope [\SI{}{\nano\mol\per\second}]},
					ymajorgrids,
					% zerofill the y-axis ticks
					xticklabel style = {
						/pgf/number format/fixed,
						/pgf/number format/fixed zerofill,
						/pgf/number format/precision = 2
					},
 					legend cell align = {left},	% left align the legend entries
					legend pos = north east
				]

		% read the data - extracted slope vs. applied potential
		%
		\pgfplotstableread[comment chars = {\%}]
		{2_Sec/generators/test_cell_applied_pot_slope_simulation.txt}
			\avgConcextremelyNormalab;

		% SIMULATION DATA FROM COMSOL
		\addplot [
					color = black,
					mark  = x,
					mark size = 3.0,
					only marks
				]
		table
		[
			x expr=\thisrowno{0},
			y expr=\thisrowno{1},
		] {\avgConcextremelyNormalab};
		\addlegendentry{COMSOL}

		% NUMERICAL FIT - polynom with degree = 10
		%
		% numerical fit:	a_n * x^n + a_(n-1) * x^(n-1) + ... + a_1 * x^1 + a_0
		%
		% add the plot (directly from the .py program)
		\addplot[
					solid,
					samples = \samples,
					no marks,
					domain = -0.80:-0.60
				]
				{-325529.693360*x^10+-573607.060870*x^9+66757.565960*x^8+374041.732600*x^7+-225999.567900*x^6+-171925.347870*x^5+373596.771390*x^4+235387.314490*x^3+-132137.309440*x^2+-138497.155330*x^1+-30445.259211};
				\addlegendentry{Polynomial deg(10)}

	\end{axis}
\end{tikzpicture}
