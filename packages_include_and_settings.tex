% load packages (\usepackage)
% set typesetting variables / settings
% debug settings (show over-/underfull boxes), show pagemargins

%%%%%%%%%%%%%%%%%%%%%%%%%%%%%%%%%% DEBUGGING %%%%%%%%%%%%%%%%%%%%%%%%%%%%%%%%%%%
%\usepackage{showframe}		% show page margins
%\overfullrule = 5pt		% mark overfull boxes on the right side (black bar)
%%%%%%%%%%%%%%%%%%%%%%%%%%%%%%%%%% DEBUGGING %%%%%%%%%%%%%%%%%%%%%%%%%%%%%%%%%%%

\usepackage{etex}
\pdfminorversion		= 5	% PDF (minor)version
\pdfcompresslevel		= 9	% compresses streams as page descriptions
\pdfobjcompresslevel	= 2	% compresses further data structures (requires PDF-1.5.)
\usepackage{amsmath}		% Extrasymbols - split environment
%\numberwithin{figure}{section}	% add number of sections to figures, e.g., Fig 1.x in sec one
%\numberwithin{table}{section}	% same as above, for tables
\usepackage{amssymb}		% Extrasymbols
\usepackage{graphicx}		% Picture
\usepackage{graphics}		% Picture
\usepackage[latin1]{inputenc}  % mutated vowels
% \usepackage{titlesec}		% change headings
\usepackage{xcolor}			% define custom colors (\definecolor)
\usepackage[skip = 5pt, hypcap = false, font = small]{caption}
	%font = scriptsize, footnotesize, small, normalsize, large or Large
% \captionsetup[subfloat]{captionskip=5pt}
	% distance picture to caption (=`(a)`)
% \usepackage{titleps}		% Headers und Footers bearbeiten
\usepackage[protrusion = true, expansion = true]{microtype}
\usepackage{wrapfig}		% (float images in text)
\usepackage[titletoc]{appendix}
\usepackage[percent]{overpic}
\usepackage{lmodern}
%  \usepackage{kpfonts}
% \usepackage[nottoc, notlot, notlof]{tocbibind}
\usepackage{tabularx}
%\usepackage{tocloft}		% formatting of table of contents/figues
\usepackage{epstopdf}
\usepackage[hcentering,
			bindingoffset  =  8 mm,	% binding offset inner side of page
			marginparwidth = 25 mm,	% width of margin box
			marginparsep   =  3 mm		% distance: written text <-> margin
			]{geometry}

\usepackage{setspace}
\usepackage{mathtools}		% space in cases-environment
\usepackage{bm}				% bold math
\usepackage[version = 4]{mhchem}	% chemical arrows, reactions, ...
\usepackage[T1]{fontenc}

% lettrine (drop cap -> first letter big)
% http://mirror.easyname.at/ctan/macros/latex/contrib/lettrine/doc/lettrine.pdf
\usepackage{lettrine}
\renewcommand{\LettrineTextFont}{\normalfont}
\setlength{\DefaultNindent}{0pt}					% horizontal shift of indented lines
\setlength{\DefaultFindent}{3pt}					% horizontal distance: dropcap <-> text
\renewcommand*{\DefaultLoversize}{0.7}				% size of the first letter
\setcounter{DefaultLines}{3}						% lines the drop cap takes (on the bottom)
\newcommand{\lettrineDropCapFontColor}{black!50}	% font color of the drop cap letter
\newcommand{\dropCapSecVertDist}{0mm}				% vertical distance: section <-> text

% brackets larger than \Bigg
\makeatletter
\newcommand{\vast}{\bBigg@{4}}
\newcommand{\Vast}{\bBigg@{5}}
\newcommand{\vastl}{\mathopen\vast}
\newcommand{\vastm}{\mathrel\vast}
\newcommand{\vastr}{\mathclose\vast}
\newcommand{\Vastl}{\mathopen\Vast}
\newcommand{\Vastm}{\mathrel\Vast}
\newcommand{\Vastr}{\mathclose\Vast}
\makeatother

\usepackage{tikz, tikz-3dplot, pgfplots, pgfplotstable}
\usetikzlibrary{positioning}
\usetikzlibrary{arrows.meta}
%\usepgfplotslibrary{ternary}
\usepgfplotslibrary{fillbetween}
\pgfplotsset{compat = 1.9}
% \usetikzlibrary{shadows, positioning}
% \usetikzlibrary{backgrounds}
% \usetikzlibrary{shapes, arrows, chains}	%flowcharts
\usepackage[oldvoltagedirection]{circuitikz}	% electrical circuits
%\usepackage[RPvoltages]{circuitikz}

\usepackage{verbatim}		% comment environment
	% multi-line comments (\begin{comment} Commented code \end{comment})
\usepackage[skip = 5pt, hypcap = false, font = small]{caption}
\usepackage[font = footnotesize]{subfig}
	% font = scriptsize, footnotesize, small, normalsize, large or Large
\usepackage{blindtext}					% dummy text
\usepackage{siunitx}
\usepackage{textcomp}
\usepackage{chngcntr}
% \usepackage{arydshln}					% dashed lines in tables
\usepackage{dcolumn}
\usepackage{fancyvrb}
\usepackage[automark]{scrlayer-scrpage}	% set \leftmark and \rightmark automatically by \section and \subsection
\usepackage{verbdef}					% http://ctan.org/pkg/verbdef
\usepackage{calc}						% http://ctan.org/pkg/calc
\usepackage{verbatimbox}				% verbatim in tikz node
\usepackage{float}						% figures: option H - places the figure HERE
\usepackage{longtable}					% tables over multiple pages
\usepackage{array, booktabs}
\usepackage[hidelinks, pagebackref = true, linktocpage = true]{hyperref}		% Links
% linktocpage = true ... colored page numbers instead of the names of e.g. sections in the TOC
% latex backlinks to bibliography (cited on page ... at the bibliography) -> pagebackref = true
%https://tex.stackexchange.com/questions/183702/formatting-back-references-in-bibliography-bibtex
\renewcommand*{\backref}[1]{}
\renewcommand*{\backrefalt}[4]{({\normalsize%
    \ifcase #1 Not cited.%
          \or Cited on page~#2.%
          \else Cited on pages #2.%
    \fi%
    })}
%
\hypersetup{colorlinks, linkcolor = blue, citecolor = orange, citecolor = green}	% colors of hyperref package
%\hypersetup{colorlinks = false} 													% black and white version
\counterwithin{figure}{section}			% set this after loading hyperref
\counterwithin{table}{section}			% set this after loading hyperref
%\hypersetup{pageanchor = false}
\usepackage{svg}						% load SVG files (needs Inkscape & shell-escape)
\usepackage{enumitem}					% adjust vertical spacing in enumerate environments
%\usepackage[letterspace=150]{microtype}
\usepackage[version = 4]{mhchem}		% chemical formulas
\usepackage{cite}						% multiple citation
\usepackage{etoolbox}					% toggle -> at document start (to display additional information)
\usepackage{multirow}					% multirow (merge rows in tabular environment)
\usepackage{cite}

% multi-cite pages (https://tex.stackexchange.com/questions/18910/multiple-citations-with-pages-using-biblatex)
\makeatletter
\newcommand{\citecomment}[2][]{\citen{#2}#1\citevar}
\newcommand{\citeone}[1]{\citecomment{#1}}
\newcommand{\citetwo}[2][]{\citecomment[,~#1]{#2}}
\newcommand{\citevar}{\@ifnextchar\bgroup{;~\citeone}{\@ifnextchar[{;~\citetwo}{]}}}
\newcommand{\citefirst}{\@ifnextchar\bgroup{\citeone}{\@ifnextchar[{\citetwo}{]}}}
\newcommand{\cites}{[\citefirst}
\makeatother
%Which you can use like this: \cites[p.1]{ref1}{ref2}[chap.3]{ref3}

% margin notes
\usepackage{marginnote}					% notes outside of page
% color & font size of text
\renewcommand*{\marginfont}{\color{black}\fontsize{5}{7}\selectfont}

% figures & captions
\usepackage{caption}
\captionsetup{belowskip = -10.0pt}		% set padding of figures (distance:
										% caption-text and the text following the
										% figure)
\captionsetup{format = hang}			% indent figure multi-line caption globally

%%%%%%%%%%%%%%%%
%%% TOC, LOF %%%
%%%%%%%%%%%%%%%%
% change the (vertical) distance between entries in the list of figures
\addtocontents{lof}{\linespread{1.25}\selectfont}	% set the vertical distance between entries in the list of figs
\addtocontents{lot}{\linespread{1.25}\selectfont}	% set the vertical distance between entries in the list of tables
\addtocontents{toc}{\linespread{1.06}\selectfont}	% set the vertical distance between entries in the table of contents

% vertical distance control in TOC (between entries)
%\DeclareTOCStyleEntry[beforeskip=.5cm]{section}{section}
\DeclareTOCStyleEntry[beforeskip=.1cm]{default}{subsection}
\DeclareTOCStyleEntry[beforeskip=.075cm]{default}{subsubsection}
\DeclareTOCStyleEntry[beforeskip=.05cm]{default}{paragraph}

% distance TOC-title and first entry below
% \setlength\cftaftertoctitleskip{50pt}
%%%%%%%%%%%%%%%%

\showboxdepth = \maxdimen
\showboxbreadth = \maxdimen

\newcommand{\hopxs}{\hspace{1px}}		% horizontal spacing of 1px

% equations with outlined box
\usepackage{empheq}
\setlength\fboxsep{0.1cm}
\newcommand*\widefbox[1]{\fbox{\hspace{1em}#1\hspace{1em}}}
% equations with outlined box

\renewcommand\appendixtocname{Appendix}  % rename appendix in the TOC
\renewcommand\appendixpagename{Appendix} % rename appendix at the end

% definition of custom colors
\definecolor{changes}{RGB}{200, 0, 0}
%\definecolor{myOrange}{RGB}{1, 0.5, 0}

\newcolumntype{L}[1]{>{\raggedright\arraybackslash}p{#1}}
\newcolumntype{C}[1]{>{\centering\arraybackslash}p{#1}}
\newcolumntype{R}[1]{>{\raggedleft\arraybackslash}p{#1}}

\usetikzlibrary{patterns}				% filled surfaces with tikz
\usetikzlibrary{backgrounds}			% TIKZ - stuff
\usetikzlibrary{arrows}

% \usepackage{geometry}  % edit space on the pages
%\geometry{tmargin=25mm,bmargin=25mm,lmargin=15mm,rmargin=20mm} % set margins

\DeclareMathAlphabet{\mathpzc}{OT1}{pzc}{m}{it}


% include section numbers in the equation numbering (e.g. (1.10))
\numberwithin{equation}{section}

% Header / Footer settings
%
% scrlayer-scrpage settings
% \ihead{\leftmark} % section at the inner side
% \ohead{\ifstr{\leftmark}{\rightmark}{}{\rightmark}} % subsection at the outer side

\renewcommand*{\subsectionmarkformat}{}		% remove numbers, e.g., 2.3 of
											% 2.3 Three?Electrode Setup
											% on the right side of uneven pages

% \newcommand*{\appendixmore}{%
%   \renewcommand\sectionmark[1]{%
% 	\ihead{\leftmark} % section at the inner side
%     \ohead{\ifstr{\leftmark}{\rightmark}{}{\rightmark}}
% }}

% font of sections, TOC ...
\addtokomafont{disposition}{\rmfamily}

%\fussy
%\sloppy
\hfuzz = 5.0pt	% ignore Overfull hbox warnings with box sizes below 5pt

% rename Figure, table of contents, bibliography, ...
\renewcommand{\figurename}{Figure}
\renewcommand{\listfigurename}{List of Figures}
\renewcommand{\contentsname}{\Huge Contents}
\renewcommand{\refname}{Bibliography}

%\captionsetup[subfigure]{position=bottom, labelfont=small}
%\captionsetup[figure]{labelfont=it,textfont={bf,it}}

% change font size of section, subsection, ...
% \setkomafont{section}{\LARGE}
\setkomafont{subsection}{\large}
%\setkomafont{subsubsection}{\tiny}
%\setkomafont{paragraph}{\large}
%\setkomafont{subparagraph}{\large}

% change spacings after a \section
\RedeclareSectionCommand[
% 							afterindent = false,
% 							beforeskip = \baselineskip,
							afterskip = .75\baselineskip
						]
						{section}

\setcounter{secnumdepth}{4}	% number depth of sections (e.g. 3 = max 1.1.1)
\setcounter{tocdepth}{4}	% maximum depth to be listed in the TOC

% \newcommand{\paragraphbrk}[1]{\paragraph{#1}\mbox{}\\}	% linebreak after \paragraphbrk{}

% hide subsubsection in the TOC while remaining the counter in the document
\newcommand{\hiddensubsubsection}[1]{
    \stepcounter{subsubsection}
    \subsubsection*{\bfseries\normalsize\arabic{section}.\arabic{subsection}.\arabic{subsubsection}\hspace{1ex}{#1}}
}

%\setlength{\belowcaptionskip}{25pt plus 3pt minus 2pt} % space between
														% picture/caption

% edit paragraphs (1.1.1.1 xyz)
%
% reduce vertical distance around the paragraphs
\setlength\parskip{1em plus 0.1em minus 0.3em}

% create a custom paragraph with an including automatically inserted linebreak
\newcommand{\lnbrkparagraph}[1]{\paragraph{#1}\mbox{}\\}
\newcommand{\lnbrkparagraphPostSkipDist}{-2.5mm}		% reduce distance: lnbrk <-> text bottom

% create a custom subparagraph with an including automatically inserted linebreak
\newcommand{\lnbrksubparagraph}[1]{\subparagraph{#1}\mbox{}\\}
\newcommand{\lnbrksubparagraphPostSkipDist}{-2.5mm}		% reduce distance: lnbrk <-> text bottom


% hide entries in TOC -> used in appendix: license
\newcommand{\nocontentsline}[3]{}
\newcommand{\tocless}[2]{\bgroup\let\addcontentsline = \nocontentsline#1{#2}\egroup}

% bold equation reference (first line of caption which includes an equation reference)
\newcommand{\beqref}[1]{\textbf{(\ref{#1})}}

% change the formatting of sections (Section 1, next line -> text)
\renewcommand{\thesection}{\arabic{section}}
\renewcommand{\thesubsection}{\arabic{section}.\arabic{subsection}}
\makeatletter
	\def\@seccntformat#1{\csname #1ignore\expandafter\endcsname\csname the#1\endcsname\quad}
	\let\sectionignore\@gobbletwo
	\let\latex@numberline\numberline
	\def\numberline#1{\if\relax#1\relax\else\latex@numberline{#1}\fi}
\makeatother
%%% sec 1 - 4 %%%
\newcommand{\customSecPreDist}{2.5mm}	% distance Section 1 <-> heading
\newcommand{\customSecPostDist}{10mm}	% distance heading <-> text
\newcommand{\customSecFontSize}{17}		% fontsize of ``Section 1''
%
\newcommand{\customSecFontSizeAdd}{20.0pt}		% fontsize of the text after the section
\newcommand{\customSecLineDistAdd}{25pt}		% linespacing of the text after the section
%
%%% appendix %%%
\newcommand{\customSecPreDistAppdx}{0mm}	% distance Appendix A <-> heading
\newcommand{\customSecPostDistAppdx}{2.5mm}	% distance heading <-> text

% %%%%%%%%%%%%%%%%%%%%%%%%%%%%%%%%%%%%%%%%%%%%%%%%%%%%%%%%%%%%%%%%%
% %%%%%%%% ADD SECTION NAMES TO GROUP FIGS IN LIST OF FIGS %%%%%%%%
% %%%%%%%%%%%%%%%%%%%%%%%%%%%%%%%%%%%%%%%%%%%%%%%%%%%%%%%%%%%%%%%%%
% list of figures -> add x section, group the listed figures
% group figs in list of figures (add space when new 'block' of figs from a section starts)
%
% retrieve the section name
% \renewcommand\sectionmark[1]{%
% 	\def\sectionname{#1}%
% 	\markright{\thesection #1}
% }

%\let\Sectionmark\sectionmark
%\def\sectionmark#1{\def\Sectionname{#1}\Sectionmark{#1}}
% \bfseries Chapter~\thechapter~\Chaptername \vskip10pt}
%
% manipulate the listed figs
%\preto\figure{%
%  \ifnum\value{figure}=0\addtocontents{lof}{{\vskip10pt \bfseries \thesection~\Sectionname \vskip2.5pt}}\fi
%}
%
% May be performed with the commented code above, but since this breaks the appendix entries, the entries in the lof
% was done manually using '\addtocontents{lof}{\vspace{\lofSecPreDist}\protect\subsubsection*{Section~\thesection~--~...}\vspace{\lofSecPostDist}}'
% at the beginnint of each section
%
% distances before & after the 'section-dividers (x section dividing blocks of figs.)' in the list of figures
% section dividers (still uses subsec)
\newcommand{\lofSecPreDist}{-3mm}		% distance before 'x section - ....'
\newcommand{\lofSecPostDist}{-2mm}		% distance after  'x section - ....'
\newcommand{\lofSecIndent}{0.0mm}		% indention of the lof subsec-entry
%
% subsection dividers (smaller font size)
\newcommand{\lofSubSecPreDist}{-3.0mm}	% distance before 'x subsection - ....'
\newcommand{\lofSubSecPostDist}{-2.0mm}	% distance after  'x subsection - ....'
\newcommand{\lofSubSecIndent}{2.0mm}	% indention of the lof subsubsec-entry
\newcommand{\lofSubSecFontA}{9.0}		% font size parameter 1
\newcommand{\lofSubSecFontB}{11}		% font size parameter 2

% footnote control (left aligned)
\deffootnote[1em]{1em}{1em}{%
	\textsuperscript{\makebox[1em][l]{\thefootnotemark}}}
